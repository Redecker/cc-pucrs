\documentclass[12pt]{article}

\usepackage{sbc-template}

\usepackage{graphicx,url}

\usepackage[brazil]{babel}   
%\usepackage[latin1]{inputenc}  
\usepackage[utf8]{inputenc}  
% UTF-8 encoding is recommended by ShareLaTex

     
\sloppy

\title{Exercício 3 - Laboratório de Redes de Computadores}

\author{Matheus S. Redecker\inst{1}}


\address{Pontifícia Universidade Católica do Rio Grande do Sul - PUCRS
  \email{  matheus.redecker@acad.pucrs.br}
}

\begin{document} 

\maketitle

\section{Exercícios}
-
    \subitem{1)} sudo ping -b -f 10.32.143.255, com isso enviamos pacotes ICMP echo request em broadcast e em tese podemos derrubar a rede, pois o trafego de pacotes é muito grande. 
    
    \subitem{2)} Está sendo gerado mensagens ICMP echo request mas não obtêm resposta. Para resolver o problema devemos habilitar a opção de forward na maquina que nós colocamos como rota, para essa maquina repassar o pacote para o roteador, assim o host manda uma mensagem de ICMP Redirect e a partir de agora as mensagens são enviadas diretamente para o endereço e não passam mais pelo host especificado. 
    
    \subitem{3)} A maquina que está atacando fica mandando pacotes ARPS dizendo para a maquina atacada que o MAC do roteador é o MAC de quem está atacando sendo assim todo o trafego do IP atacado é passado para a maquina que está atacando e a maquina atacada não recebe mais nada do roteador. 

\end{document}
