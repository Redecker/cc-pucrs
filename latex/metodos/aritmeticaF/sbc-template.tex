\documentclass[12pt]{article}

\usepackage{sbc-template}

\usepackage{graphicx,url}

\usepackage[brazil]{babel}   
%\usepackage[latin1]{inputenc}  
\usepackage[utf8]{inputenc}  
% UTF-8 encoding is recommended by ShareLaTex

\usepackage{minted}
     
\sloppy

\title{Exercícios sobre aritmética F - Métodos Computacionais}

\author{Matheus S. Redecker\inst{1}}


\address{Pontifícia Universidade Católica do Rio Grande do Sul - PUCRS
  \email{  matheus.redecker@acad.pucrs.br}
}

\begin{document} 

\maketitle


\section{Exercícios}

\subitem{1)} 
    $ 2 * x^{3} - 3 * x^{2} - 4x + 6 $ com $x = 1,4 $ é igual a $0,008$
    \subsubitem{a)} \\ $ 2 * (1,4^{3}) - 3 * (1,4^{2}) - 4 * 1,4 + 6  $ = \\ $ 2 * (2,744) - 3 * (1,96) - 5,6 + 6  $ = \\ $2 * (2,74) - 3 * (1,96) - 5,6 + 6  $ = \\ $5,48 - 5,88 - 5,6 + 6 $ = \\ $0$
    \subsubitem{b)} \\ $ 2 * (1,4^{3}) - 3 * (1,4^{2}) - 4 * 1,4 + 6  $ = \\ $ 2 * (2,744) - 3 * (1,96) - 5,6 + 6  $ = \\ $2 * (2,75) - 3 * (1,96) - 5,6 + 6  $ = \\ $5,5 - 5,88 - 5,6 + 6 $ = \\ $0,02$\\
A avaliação mais exata é a do arredondamento por truncamento pois é o resultando mais perto do valor real que foi 0,008.
\subitem{2)} O epision de maquina é o menor numero que somado a 1 resulta em um numero diferente de 1. Uma implementação da tecnica usada para descobrir o valor está apresentada abaixo na linguaguem python, o resultado obtido foi o valor $2.22044604925 e^{-16} $
\begin{minted}{python}
	eps =1.0
	while (eps+1) != 1:
		eps = eps/2
\end{minted} 

\subitem{3)} Sim, pois dependendo da operação que for feita primeiro pode ou não ter overflow, para melhor exemplificar segue o exemplo abaixo: \\
Considere o sitema de ponto flutuante F = F(10,3,-6,7): \\
$(a * b) * c $ = $(2 * 5) * 0,5$ = $(10)* 0,5$ - aqui temos um overflow \\
Já se invertermos as posições: \\
$a * (b * c)$ = $2 * (5 * 0,5)$ = $2 * (2,5)$ = $ 5 $ - o overflow não ocorre \\
Sendo assim a ordem que são feitas as operações pode ocasionar um overflow.
\newpage
\subitem{4)} 
    \subsubitem{a)} Dois numeros muito grandes somados provavelmente ocorrerá um overflow.
    \subsubitem{b)} Dois numeros proximos muito grandes provavelmente ocorrerá um underflow.
    \subsubitem{c)} Duas grandezas muito desproporcionais multiplicadas provavelmente ocorrerá um overflow.
    \subsubitem{d)} -
\subitem{5)} 99,996\%

\end{document}
