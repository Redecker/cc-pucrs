\documentclass[12pt]{article}

\usepackage{sbc-template}

\usepackage{graphicx,url}

\usepackage[brazil]{babel}   
%\usepackage[latin1]{inputenc}  
\usepackage[utf8]{inputenc}  
% UTF-8 encoding is recommended by ShareLaTex

     
\sloppy

\title{Final Assignment}

\author{Matheus S. Redecker\inst{1} }

\address{Pontifícia Universidade Católica do Rio Grande do Sul
 \email{matheus.redecker@acad.pucrs.br}
}


\begin{document} 

\maketitle

\section{Chapter 5 - What are the current trends in software platforms?}
Are trends in software platforms:
\begin{itemize}
    \item Open source software is software produced by a community of several hundred thousand programmers around the world. Open source software is free and can be modified by users. Works derived from the original code must also be free, and the software can be redistributed by the user without additional licensing. 
    \item Linux was created by the Finnish programmer Linus Torvalds and first posted on the Internet in August 1991. Linux applications are embedded in cell phones, smart phones, notebooks, and consumer electronics. Linux is available in free versions downloadable from the Internet or in low-cost commercial versions that include tools and support from vendors such as Red Hat. The rise of open source software, particularly Linux and the applications it supports, has profound implications for corporate software platforms: cost reduction, reliability and resilience, and integration, because Linux works on all the major hardware platforms from mainframes to servers to clients.
    \item Java is an operating system-independent, processor-independent, object oriented programming language that has become the leading interactive environment for the Web. The Java platform has migrated into cellular phones, smart phones, automobiles, music players, game machines, and finally, into set-top cable television systems serving interactive content and pay-per-view services. Java software is designed to run on any computer or computing device, regardless of the specific microprocessor or operating system the device uses. 
    \item Ajax allows a client and server to exchange small pieces of data behind the scene so that an entire Web page does not have to be reloaded each time the user requests a change. 
    \item Web services refer to a set of loosely coupled software components that exchange information with each other using universal Web communication standards and languages. They can exchange information between two different systems regardless of the operating systems or programming languages on which the systems are based. 
    \item Cloud-based software and the data it uses are hosted on powerful servers in massive data centers, and can be accessed with an Internet connection and standard Web browser. 
\end{itemize}

\section{Chapter 8 - Why are information systems vulnerable to destruction, error, and abuse?}
When large amounts of data are stored in electronic form, they are vulnerable to many more kinds of threats than when they existed in manual form. Computers that are constantly connected to the Internet are more open to penetration by outsiders because they use fixed Internet addresses where they can be easily identified.
Vulnerability has also increased from widespread use of e-mail, instant messaging, and peer-to-peer file-sharing programs. E-mail may contain attachments that serve as springboards for malicious software or unauthorized access to internal corporate systems. Employees may use e-mail messages to transmit valuable trade secrets, financial data, or confidential customer information to unauthorized recipients. Malicious software programs are referred to as malware and include a variety of threats, such as computer viruses, worms, and Trojan horses. A computer virus is a rogue software program that attaches itself to other software programs or data files in order to be executed, usually without user knowledge or permission. Most recent attacks have come from worms, which are independent computer programs that copy themselves from one computer to other computers over a network. Worms destroy data and programs as well as disrupt or even halt the operation of computer networks. Worms and viruses are often spread over the Internet from files of downloaded software, from files attached to e-mail transmissions, or from compromised e-mail messages or instant messaging. A hacker is an individual who intends to gain unauthorized access to a computer system. In a denial-of-service (DoS) attack, hackers flood a network server or Web server with many thousands of false communications or requests for services to crash the network. The network receives so many queries that it cannot keep up with them and is thus unavailable to service legitimate requests. A distributed denial-of-service (DDoS) attack uses numerous computers to inundate and overwhelm the network from numerous launch points. Click fraud occurs when an individual or computer program fraudulently clicks on an online ad without any intention of learning more about the advertiser or making a purchase. Malicious intruders seeking system access sometimes trick employees into revealing their passwords by pretending to be legitimate members of the company in need of information. This practice is called social engineering. Software errors pose a constant threat to information systems, causing untold losses in productivity. Growing complexity and size of software programs, coupled with demands for timely delivery to markets, have contributed to an increase in software flaws or vulnerabilities

\section{Chapter 9 - How do customer relationship management systems help firms achieve customer intimacy?}
Customer relationship management (CRM) systems capture and integrate customer data from all over the organization, consolidate the data, analyze the data, and then distribute the results to various systems and customer touch points across the enterprise. A touch point  is a method of interaction with the customer, such as telephone, e-mail, customer service desk, conventional mail, Web site, wireless device, or retail store. Well-designed CRM systems provide a single enterprise view of customers that is useful for improving both sales and customer service. Such systems likewise provide customers with a single view of the company regardless of what touch point the customer uses. Good CRM systems provide data and analytical tools for answering questions related to the customers. Firms use the answers to these questions to acquire new customers, provide better service and support to existing customers, customize their offerings more precisely to customer preferences, and provide ongoing value to retain profitable customers. Customer service modules in CRM systems provide information and tools to increase the efficiency of call centers, help desks, and customer support staff. They have capabilities for assigning and managing customer service requests. One such capability is an appointment or advice telephone line: When a customer calls a standard phone number, the system routes the call to the correct service person, who inputs information about that customer into the system only once. Once the customer’s data are in the system, any service representative can handle the customer relationship. Improved access to consistent and accurate customer  information help call centers handle more calls per day and decrease the duration of each call. Thus, call centers and customer service groups achieve greater productivity, reduced transaction time, and higher quality of service at lower cost. The customer is happier because he or she spends less time on the phone restating his or her problem to customer service representatives. Companies with effective customer relationship management systems realize many benefits, including increased customer satisfaction, reduced direct marketing costs, more effective marketing, and lower costs for customer acquisition and retention. Information from CRM systems increases sales revenue by identifying the most profitable customers and segments for focused marketing and cross-selling.

\section{Chapter 10 - What issues must be addressed when building an e-commerce Web site?}
Building a successful e-commerce site requires a keen understanding of business, technology, and social issues.
When you are building an e-commerce web site you will have to bring together a team of individuals who possess the skill sets needed to build and manage a successful e-commerce site. This team will make the key decisions about technology, site design, and social and information policies that will be applied at your site. The entire site development effort must be closely managed if you hope to avoid the disasters that have occurred at some firms. You will also need to make decisions about your site’s hardware, software, and telecommunications infrastructure. The demands of your customers should drive your choices of technology. Your customers will want technology that enables them to find what they want easily, view the product, purchase the product, and then receive the product from your warehouses quickly. You will also have to carefully consider your site’s design. Once you have identified the key decision areas, you will need to think about a plan for the project. Your planning should identify the specific business objectives for your site, and then develop a list of system functionalities and information requirements. There are many choices for building and maintaining Web sites. Much depends on how much money you are willing to spend. Choices range from outsourcing the entire Web site development to an external vendor to building everything yourself (in-house). Most businesses choose to outsource hosting and pay a company to host their Web site, which means that the hosting company is responsible for ensuring the site is “live” or accessible, 24 hours a day.

\section{Chapter 11 - What are the business benefits of using intelligent techniques for knowledge management?}
Artificial intelligence and database technology provide a number of intelligent techniques that organizations can use to capture individual and collective knowledge and to extend their knowledge base. Expert systems, case-based reasoning, and fuzzy logic are used for capturing tacit knowledge. Neural networks and data mining are used for knowledge discovery. They can discover underlying patterns, categories, and behaviors in large data sets that could not be discovered by managers alone or simply through experience. Genetic algorithms are used for generating solutions to problems that are too large and complex for human beings to analyze on their own. Intelligent agents can automate routine tasks to help firms search for and filter information for use in electronic commerce, supply chain management, and other activities. Expert systems are an intelligent technique for capturing tacit knowledge in a very specific and limited domain of human expertise. These systems capture the knowledge of skilled employees in the form of a set of rules in a software system that can be used by others in the organization. The set of rules in the expert system adds to the memory, or stored learning, of the firm. Human knowledge must be modeled or represented in a way that a computer can process. Expert systems model human knowledge as a set of rules that collectively are called the knowledge base. Expert systems provide businesses with an array of benefits including improved decisions, reduced errors, reduced costs, reduced training time, and higher levels of quality and service. Expert systems primarily capture the tacit knowledge of individual experts, but organizations also have collective knowledge and expertise that they have built up over the years. This organizational knowledge can be captured and stored using case-based reasoning. In case-based reasoning (CBR), descriptions of past experiences of human specialists, represented as cases, are stored in a database for later retrieval when the user encounters a new case with similar parameters. 

\section{Chapter 12 - What are the different types of decisions and how does the decision-making process work?}
Decisions are classified as structured, semistructured, and unstructured. Unstructured decisions are those in which the decision maker must provide judgment, evaluation, and insight to solve the problem. Each of these decisions is novel, important, and nonroutine, and there is no well-understood or agreed-on procedure for making them. Structured decisions are repetitive and routine, and they involve a definite procedure for handling them so that they do not have to be treated each time as if they were new. Many decisions have elements of both types of decisions and are semistructured, where only part of the problem has a clear-cut answer provided by an accepted procedure. Making a decision is a multistep process. There are four different stages in decision making: intelligence, design, choice, and implementation. Intelligence consists of discovering, identifying, and understanding the problems occurring in the organization. Design involves identifying and exploring various solutions to the problem. Choice consists of choosing among solution alternatives. And implementation involves making the chosen alternative work and continuing to monitor how well the solution is working. 
\end{document}
